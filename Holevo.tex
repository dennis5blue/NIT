\begin{subsection}{Holevo Bound}
After introducing some basic ideas of quantum information theory, we are now interested in how much information can be carried in a quantum system.
Recall that in the classical world, the amount of mutual information is bounded by the following equation:
\begin{equation}\label{eq:cBound}
I(X;Y) \leq H(X).
\end{equation}
However, in the quantum world, equation~\eqref{eq:cBound} will be further bounded by a lower limit, say {\it{Holevo bound}}.
We now give the following theorem together with an intutive example, and the complete proof will be written in section~\ref{sec:HolevoPf}.

\textbf{Theorem 1 (Holevo Bound)}
Assume that a sender Alice obtains a message $X$ according to some probability distribution $p_1,p_2,\cdots p_n$. She encodes the message $X$ in a quantum state $\rho_x$ and sends the quantum state to Bob. Bob receives $\{p_x,\rho_x\}$ and wants to obtain as much information as he can, then the amount of information is bounded by the following equation:
\begin{equation}\label{eq:HolevoBound}
I(X;Y) \leq H(\rho) - \underset{x}{\sum} p_xH(\rho_x) \leq H(X).
\end{equation}

We now give an example to show how {\it{Holevo bound}} works.
Suppose that Alice transmits $|0\rangle$ if $X=0$ and transmits $\cos \theta |0\rangle + \sin \theta |1\rangle$ if $X=1$, where $X=\{0,1\}$ follows a uniform distribution.
Then we can have two pure state density matrices according to the transmitted state when $X=0$ or $X=1$ as follows:
\begin{align}\label{eq:2densityMatrix}
&\rho_1 = |0\rangle \langle 0| = \begin{pmatrix} 1 &0 \\ 0 &0 \end{pmatrix}, \nonumber \\
&\rho_2 = \begin{pmatrix} \cos \theta \\ \sin \theta \end{pmatrix}
         \begin{pmatrix} \cos \theta & \sin \theta \end{pmatrix} = 
         \begin{pmatrix} \cos^2 \theta &\sin \theta \cos \theta \\
         \sin \theta \cos \theta &\sin^2 \theta \end{pmatrix}.
\end{align}
Note that the two density matrices in~\eqref{eq:2densityMatrix} both has eigenvalues $0$ and $1$; therefore, $H(\rho_1) = H(\rho_2) = -1\log 1 - 0\log 0 = 0$.
However, the density matrix of mixed state $\rho_1$ and $\rho_2$ is:
\begin{align}\label{eq:mixed2State}
\rho = \sum_i p_i \rho_i = \frac{1}{2}(\rho_1 + \rho_2) = 
\begin{pmatrix} 1+\cos^2 \theta &\sin \theta \cos \theta \\
 \sin \theta \cos \theta &\sin^2 \theta \end{pmatrix}.
\end{align}
The density matrix in~\eqref{eq:mixed2State} has eigenvalues $\frac{1+\cos \theta}{2}$ and $\frac{1-\cos \theta}{2}$, and hence it quantum entropy is written as: 
\begin{equation}
H(\rho) = -\frac{1+\cos \theta}{2} \log \frac{1+\cos \theta}{2} - \frac{1-\cos \theta}{2} \log \frac{1-\cos \theta}{2},
\end{equation}
which will be maximized when $\theta = \pi /2$ and the maximum equals $1$.
Therefore, by~\eqref{eq:HolevoBound}, we know that $I(X;Y) \leq H(\rho)-0 \leq 1$.
It means that Bob can obtain \textit{at most} $1$ bit of information when $\theta = \pi /2$.
This result is very important, since it essentially says that one qubit can contain at most one classical bit information.
Therefore, we can learn that quantum information does not compress classical information.
\end{subsection}
