\subsection{Entropy of Quantum States}
Recall that the Shannon entropy measures the uncertainty of a classical random variable, we should take the expectation of the self information over all possible realizations.
That is,
\begin{equation}
H(X) = -\sum_i p_i \log p_i.
\end{equation}
The von Neumann entropy (or say quantum entropy) generalize the definition of Shannon entropy to quantum states, whose uncertainty can be measured by its density matrix:
\begin{equation}
H(\rho) \equiv - \text{Tr} \{ \rho \log \rho \} = -\sum_{i} \lambda_i \log \lambda_i,
\end{equation}
where the first $\log$ is the logarithm of a matrix (a matrix $B$ is the logarithm of matrix $A$ if $e^B = A$).
Since the density matrix $\rho$ can be represented by the form of eigenvalues decomposition $\rho = \sum_{i} \lambda_i |e_j\rangle \langle e_j|$; therefore, $\log \rho = \sum_{i} \log \lambda_i |e_j\rangle \langle e_j|$, which means that $\text{Tr} \{ \rho \log \rho \} = \sum_{i} \lambda_i \log \lambda_i$.

Note that the quantum entropy of a density matrix is the Shannon entropy of its eigenvalues.
Therefore, it shares many common properties with the Shannon entropy and we will show those properties in the following subsections.

\subsection{Properties of Quantum Entropy}
In this section, we discuss several mathematical properties of the quantum entropy, which includes: positivity, its minimum value, its maximum value, and concavity.

\textbf{Property 1 (Positivity)} The quantum entropy $H(\rho)$ is non-negative for any density operator $\rho$:
\begin{align}
H(\rho) \geq 0.
\end{align}
\textit{Proof.} We observe that quantum entropy can be expressed as  $-\sum_{i} \lambda_i \log \lambda_i$. Due to the $\log$ function is always greater or equal to $0$, the quantum entropy is also always greater or equal to 0.

\textbf{Property 2 (Minimum Value)}
The minimum value of quantum entropy is zero, and we can achieve it as the density operator is in a pure state; that is, the quantum state is deterministic. \\
\textit{Proof.} The minimum value occurs when a density operator are distributed with all the mass on one value and zero on the others, so that the density operator is rank one and corresponds to a pure state.

\textbf{Property 3 (Maximum Value)}
The maximum value of quantum entropy is $\log D$, where $D$ is the dimension of the system. \\
\textit{Proof.} Similarly to classical case, quantum entropy is maximum if and only if the eigenvalues $\lambda$ is uniformly distributed and it leads entropy to be:
\begin{align}
H(\rho) \equiv -\sum_{i} \lambda_i \log \lambda_i = -D\frac{1}{D} \log(\frac{1}{D})=\log D.
\end{align}

\textbf{Property 4 (Concavity)}
The quantum entropy is concave in the density matrix, $H(\rho) \geq \sum_x p_X(x) H(\rho_x)$, where $\rho = \sum_x p_X(x)\rho_x$. \\
\textit{Proof.} Define a joint state of $AB$ by:
\begin{align}
\rho^{AB} \equiv \sum_i p_i \rho_i \otimes |i\rangle \langle i|.
\end{align}
Note that for the density matrix $\rho^{AB}$ we have:
\begin{align}
&H(A)=H(\sum_i p_i \rho_i) \nonumber \\
&H(B)=H(\sum_i p_i |i \rangle \langle i|)=H(p_i) \nonumber \\
&H(A,B)=H(p_i)+\sum_i p_i H(\rho_i).
\end{align}
Note that $H(A,B) \leq H(A)+H(B)$, and hence we can obtain
\begin{align}
\sum_i p_i H(\rho_i) \leq H(\sum_i p_i \rho_i),
\end{align}
which finishes the proof of concavity.

\subsection{Other Definitions of Quantum Information}
The quantum entropy also has some interesting properties.
Before introducing them, let us give some definitions of quantum information: \\
\begin{itemize}
	\item {Joint Quantum Entropy} $H(AB) \equiv -\text{Tr}\{\rho^{AB} \log \rho^{AB}\}$.
	\item {Conditional Quantum Entropy} $H(A|B) = H(AB) - H(B)$.
	%\item {\color{red}Coherent Information} $I(A \rangle B) \equiv H(B) - H(AB)$.
	\item {Quantum Mutual Information} $I(A;B) \equiv H(A)+H(B)-H(AB)$.
	\item {Conditional Quantum Mutual Information} $I(A;B|C) \equiv H(A|C)+H(B|C)-H(AB|C)$.
\end{itemize}
Note that the conditional quantum mutual information is always non-negative, that is:
\begin{align}
I(A;B|C) \geq 0.
\end{align}
This equation is equivalent to the strong subadditivity inequality which will be proved in Section~\ref{StrSubAdd}.

\subsection{Conditional Quantum Entropy}
Unlike the classical conditional entropy, the quantum entropy can be negative.
This is true even the quantum entropy of single variable is never negative.
For example, in the phenomenon of entanglement, we have the joint state to be a pure state while the marginal states are mixed states.
That is,
\begin{align}
& |\psi_{AB} \rangle = \frac{1}{\sqrt{2}}(|00\rangle + |11\rangle), \nonumber \\
& \rho_{AB}= |\psi_{AB} \rangle \langle \psi_{AB}|, \nonumber \\
& \rho_A=\frac{1}{2}(|0\rangle \langle 0| + |1\rangle \langle 1|).
\end{align}
Therefore, the joint quantum entropy $H(AB) = 0$ while the marginal quantum entropy $H(A) = H(B) = 1$.
Hence by the definition of conditional quantum entropy, we can have the following results:
\begin{align}
&I(A;B)=2, \nonumber \\
&H(A|B)=-1. 
\end{align}
The informational statement is that we can sometimes be more certain about the joint state of a quantum system than we can be about any one of its individual parts, and this is the reason that conditional  quantum entropy can be negative.

\subsection{Coherent Information}
Negativity of the conditional quantum entropy is so important in quantum information theory that we even have an information quantity and a special notation to donate the negative of the conditional quantum entropy.
Therefore, we show the definition of the \textit{coherent information} $I(A\rangle B)_{\rho}$ of a bipartite state $\rho^{AB}$ as follows:
\begin{align}
I(A\rangle B) \equiv H(B)-H(AB).
\end{align}
We can see that the quantity is the negative of the conditional quantum entropy.
Therefore, the coherent information will be positive when the related conditional entropy is negative.
%It is very important when we further discuss the quantum data processing and the quantum channel capacity.