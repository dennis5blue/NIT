\subsection{Entropy of Quantum States}
Recall that the Shannon Entropy measures the uncertainty of a classical random variable, we should take the expectation of the self information over all possible realizations:
\begin{equation*}
H(X) = -\sum_i p_i \log p_i.
\end{equation*}
The Von Neumann Entropy generalize the definition of Shannon entropy to quantum states, whose uncertainty can be measured by its density matrix:
\begin{equation*}
H(\rho) \equiv - \text{Tr} \{ \rho \log \rho \} = -\sum_{i} \lambda_i \log \lambda_i.
\end{equation*}
Note that the von Neumann entropy of a density matrix is the Shannon entropy of its eigenvalues. Therefore, the von Neumann entropy shares many properties with the Shannon entropy.

\subsection{Properties of  Quantum Entropy}
In this section, we discuss several mathematical properties of the quantum entropy: positivity, its minimum value, its maximum value, its invariance under unitaries, and concavity.

\textbf{Property 1 (Positivity)} The von Neumann entropy $H(\rho)$ is non-negative for any density operator $\rho$:
\begin{align*}
H(\rho) \geq 0.
\end{align*}
\textit{Proof.} We observe that von Neumann entropy can be expressed as  $-\sum_{i} \lambda_i \log \lambda_i$. Due to the log function always larger than 0, the von Neumann entropy is also larger than 0.

\textbf{Property 2 (Minimum Value)}
The minimum value of von Neumann entropy is zero, and we can achieve it as the density operator is a pure state; that is, the quantum state is deterministic. \\
\textit{Proof.} The minimum value occurs when a density operator are distributed with all the mass on one value and zero on the others, so that the density operator is rank one and corresponds to a pure state.

\textbf{Property 3 (Maximum Value)}
The maximum value of von Neumann entropy is $logD$, where $D$ is the dimension of the system. \\
\textit{Proof.} Similarly to classical case, von Neumann entropy is maximum iff eigen value $\lambda$ is uniformly distributed and it leads entropy to be:
\begin{align*}
H(\rho) \equiv -\sum_{i} \lambda_i \log \lambda_i = D\frac{1}{D} log(\frac{1}{D})=logD.
\end{align*}

\textbf{Property 4 (Concavity)}
The von Neumann entropy is concave in the density matrix, $H(\rho) \geq \sum_x p_X(x) H(\rho_x)$, where $\rho = \sum_x p_X(x)\rho_x$. \\
\textit{Proof.} Define a joint state of $AB$ by:
\begin{align*}
\rho^{AB} \equiv \sum_i p_i \rho_i \otimes |i\rangle \langle i|.
\end{align*}
Note that for theensity matrix $\rho^{AB}$ we have:
\begin{align*}
H(A)=H(\sum_i p_i \rho_i) \\
H(B)=H(\sum_i p_i |i \rangle \langle i|)=H(p_i) \\
H(A,B)=H(p_i)+\sum_i p_i H(\rho_i). \\
\end{align*}
Applying the subadditivity inequality $H(A,B) \leq H(A)+H(B)$we obtain
\begin{align*}
\sum_i p_i H(\rho_i) \leq H(\sum_i p_i \rho_i),
\end{align*}
which is concavity.

\subsection{Conditional von Nuemann entropy}
Unlike the classical conditional entropy, the quantum entropy can be negative. This is true even the von Nuemann entropy of single variable is never negative. \\
For example:
Entangled Particals:
\begin{align*}
& |\psi_{AB} \rangle = \frac{1}{\sqrt{2}}(|00\rangle + |11\rangle) \\
& \rho_{AB}= |\psi_{AB} \rangle \langle \psi_{AB}| \\
& \rho_A=\frac{1}{2}(|0\rangle \langle 0| + |1\rangle \langle 1|) \\
& H(A)=1=H(B), & &I(A;B)=2,  \\
& H(AB)=0, & &H(A|B)=-1 
\end{align*}
The informational statement is tha we can sometimes be more certain about the joint state of a quantum system than we can be about any one of its individual parts, and this is the reason that conditional  quantum entropy can be negative.

\subsection{Coherent Information}
Negativity of the conditional quantum entropy is so important in quantum information theory that we even have an information quantity and a special notation to donate the negative of the conditional quantum entropy:\\
We define that the coherent inforamtion $I(A\rangle B)_{\rho}$ of a bipartite state $\rho^{AB}$ is as follows:
\begin{align*}
I(A\rangle B)_{\rho} \equiv H(B)_{\rho}-H(AB)_{\rho}
\end{align*}
We can see that the quantity is the negative of the conditional quantum entropy. It is very important when we discuss the quantum data processing and quantum channel capacity.

\subsection{Strong subadditivity}
For any quantum systems $A, B, C,$ the inequalities
\begin{align*}
H(A)+H(C) \leq H(AB)+H(BC) \\
H(ABC)+H(C) \leq H(AC)+H(BC)
\end{align*}
hold.

\emph{Proof:} Define a function $T(\rho ^{ABC})$ of density operators on the system ABC,
\begin{align*}
T(\rho ^{ABC}) &\equiv H(A)+H(C)-H(AB)-H(BC) \\
&= -H(B|A) -H(B|C) .
\end{align*}
  From the concavity of the conditional entropy we see that $T(\rho ^{ABC})$ is a convex function of $\rho ^{ABC}$ of density operators on the system $ABC$.
Let $\rho ^{ABC} = \sum_ip_i|i \rangle \langle i|$ be a spectral decomposition of $\rho ^{ABC}$.

  From the convexity of $T$, $T(\rho ^{ABC}) \leq \sum_ip_iT(|i \rangle \langle i|)$. But $T(|i \rangle \langle i|)=0$ as for a pure state $H(AB)=H(C)$ and $H(BC)=H(A)$. It follows that $T(\rho ^{ABC}) \leq 0$, and thus
\begin{align*}
H(A)+H(C)-H(AB)-H(BC) \leq 0,
\end{align*}
which is the first inequality we set out to prove.

  To obtain the second inuquality, introduce an auxiliary system $R$ purifying the system $ABC$. Then using the just-proved inuquality we have
\begin{align*}
H(R)+H(C) \leq H(RB)+H(BC).
\end{align*}
  Since ABCR is a pure state, $H(R)=H(ABC)$ and $H(RB)=H(AC)$, so becomes
\begin{align*}
H(ABC)+H(C) \leq H(AC)+H(BC).
\end{align*}
Therefore, we finish the proof.