Quantum computation has been regarded as a potential way to perform a more complicated calculation.
This is due to the reason that the quantum physics allows us to construct new types of logic gates, which might be more powerful than the classical gates.
Therefore, we are interested in the idea of quantum computation as well as the difference between classical and quantum world.
Note that in the classical world, the maximum number of different messages that can be represented in a two-state system is $2$.
However, in the quantum world, this two-state system can represent any superposition of the two states.
Besides, a quantum system also has some interesting properties, for example, the quantum entanglement.
The quantum entanglement is a physical phenomenon that partial states will interact with each other such that the whole quantum state can be given.
This will cause a fundamental difference of information theory when the information is stored in a state of a quantum system.
Those interesting parts motivates us to study the idea of quantum information theory, and hence we studied some books~\cite{CtoQ}~\cite{ElementsOfQ}~\cite{QCandQI} and tried to show what we learned in this report.

The following of this report is written as follows:
Section~\ref{sec:Background} introduces some background knowledge of the quantum information theory, including the basic element in a quantum world and how to represent the statistical state of a quantum system.
Section~\ref{sec:QuantumInfo} describes some definitions of quantum information theory and its properties, and also shows some inequalities which can be used to bound the amount of carried information.
Finally, Section~\ref{sec:Conclusion} concludes what we learned in this project, and the proof of some lemmas will be written in the appendix. 
%\cite{Quantiki}