\documentclass[12pt]{article}

\usepackage[hmargin=2.5cm, vmargin={3cm,3cm}, a4paper]{geometry}
\usepackage{CJKutf8}

\usepackage{amsmath,amssymb,mathrsfs}
\usepackage{amsthm}
\usepackage{epsfig,epsf,subfigure,graphicx,graphics}
\usepackage{url}
\usepackage{array}
\usepackage{multicol}
\usepackage{empheq}
\usepackage{enumerate}


%\newtheorem{lemma}{Lemma}
%\newtheorem{theorem}[lemma]{Theorem}
%\newtheorem{proposition}[lemma]{Proposition}
%\newtheorem{corollary}[lemma]{Corollary}
%\newtheorem{definition}[lemma]{Definition}
%\newtheorem{remark}[lemma]{Remark}
%\newtheorem{claim}[lemma]{Claim}
%\newtheorem{fact}[lemma]{Fact}
%\newtheorem{example}[lemma]{Example}
%\newtheorem{exercise}{Exercise}

\newcounter{MYtempeqncnt}

%\graphicspath{{fig_IFCCD/}{fig_IFCCE/}}
\allowdisplaybreaks

%% For Chinese %%
\usepackage{fontspec}
%\setromanfont{LiHei Pro}
%\setromanfont{BiauKai}
%\setmonofont{Courier New}
%\newfontfamily{\BK}{BiauKai}
%\newfontfamily{\LHP}{LiHei Pro}

%\XeTeXlinebreaklocale "zh"
%\XeTeXlinebreakskip = 0pt plus 1pt




%\graphicspath{{fig/}}

\usepackage{fancyhdr}
\setlength{\headheight}{12pt}
 
\pagestyle{fancyplain}

\rhead{C.Y. Song \\ C.H. Wang}
\lhead{Jan 18, 2015}
\chead{{\it From Classical to Quantum Information Theory}}
\lfoot{}
\cfoot{\thepage}
\rfoot{}

%\pagestyle{myheadings}
%\markright{}



\title{From Classical to Quantum Information Theory}
\author{R02942122 Chang-Yu Song,
        R03942073 Chun-Hsiung Wang}
\date{January 18, 2015} 
\begin{document}
\maketitle

\thispagestyle{fancyplain}

\begin{abstract}
abstract

\end{abstract}

\section{Introduction}\label{sec:Intro}
Introduction
\cite{Quantiki} \cite{CtoQ} \cite{ElementsOfQ} \cite{QCandQI}

\section{Background Knowledge}\label{sec:Background}
Background Knowledge


\section{Quantum Information and Entropy}\label{sec:QuantumInfo}
\subsection{Entropy of Quantum States}
Recall that the Shannon entropy measures the uncertainty of a classical random variable, we should take the expectation of the self information over all possible realizations:
\begin{equation}
H(X) = -\sum_i p_i \log p_i.
\end{equation}
The Von Neumann Entropy (or say quantum entropy) generalize the definition of Shannon entropy to quantum states, whose uncertainty can be measured by its density matrix:
\begin{equation}
H(\rho) \equiv - \text{Tr} \{ \rho \log \rho \} = -\sum_{i} \lambda_i \log \lambda_i,
\end{equation}
where the first $\log$ is the logarithm of a matrix (a matrix $B$ is the logarithm of matrix $A$ if $e^B = A$).
Since the density matrix $\rho$ can be represented by the form of eigenvalues decomposition $\rho = \sum_{i} \lambda_i |e_j\rangle \langle e_j|$; therefore, $\log \rho = \sum_{i} \log \lambda_i |e_j\rangle \langle e_j|$, which means that $\text{Tr} \{ \rho \log \rho \} = \sum_{i} \lambda_i \log \lambda_i$.

Note that the quantum entropy of a density matrix is the Shannon entropy of its eigenvalues.
Therefore, it shares many common properties with the Shannon entropy and we will show those properties in the following subsections.

\subsection{Properties of Quantum Entropy}
In this section, we discuss several mathematical properties of the quantum entropy, which includes: positivity, its minimum value, its maximum value, and concavity.

\textbf{Property 1 (Positivity)} The quantum entropy $H(\rho)$ is non-negative for any density operator $\rho$:
\begin{align}
H(\rho) \geq 0.
\end{align}
\textit{Proof.} We observe that quantum entropy can be expressed as  $-\sum_{i} \lambda_i \log \lambda_i$. Due to the log function is always greater than 0, the quantum entropy is also always greater than 0.

\textbf{Property 2 (Minimum Value)}
The minimum value of quantum entropy is zero, and we can achieve it as the density operator is in a pure state; that is, the quantum state is deterministic. \\
\textit{Proof.} The minimum value occurs when a density operator are distributed with all the mass on one value and zero on the others, so that the density operator is rank one and corresponds to a pure state.

\textbf{Property 3 (Maximum Value)}
The maximum value of quantum entropy is $logD$, where $D$ is the dimension of the system. \\
\textit{Proof.} Similarly to classical case, quantum entropy is maximum if and only if the eigenvalues $\lambda$ is uniformly distributed and it leads entropy to be:
\begin{align}
H(\rho) \equiv -\sum_{i} \lambda_i \log \lambda_i = D\frac{1}{D} log(\frac{1}{D})=logD.
\end{align}

\textbf{Property 4 (Concavity)}
The quantum entropy is concave in the density matrix, $H(\rho) \geq \sum_x p_X(x) H(\rho_x)$, where $\rho = \sum_x p_X(x)\rho_x$. \\
\textit{Proof.} Define a joint state of $AB$ by:
\begin{align}
\rho^{AB} \equiv \sum_i p_i \rho_i \otimes |i\rangle \langle i|.
\end{align}
Note that for the density matrix $\rho^{AB}$ we have:
\begin{align}
&H(A)=H(\sum_i p_i \rho_i) \nonumber \\
&H(B)=H(\sum_i p_i |i \rangle \langle i|)=H(p_i) \nonumber \\
&H(A,B)=H(p_i)+\sum_i p_i H(\rho_i).
\end{align}
Note that $H(A,B) \leq H(A)+H(B)$, and hence we can obtain
\begin{align}
\sum_i p_i H(\rho_i) \leq H(\sum_i p_i \rho_i),
\end{align}
which finishes the proof of concavity.

\subsection{Other Definitions of Quantum Information}
The Quantum entropy also has some interesting properties.
Before introducing them, let us give some definitions of Quantum information: \\
\begin{itemize}
	\item {Joint Quantum Entropy} $H(AB) \equiv -\text{Tr}\{\rho^{AB} \log \rho^{AB}\}$.
	\item {Conditional Quantum Entropy} $H(A|B) = H(AB) - H(B)$.
	%\item {\color{red}Coherent Information} $I(A \rangle B) \equiv H(B) - H(AB)$.
	\item {Quantum Mutual Information} $I(A;B) \equiv H(A)+H(B)-H(AB)$.
	\item {Conditional Quantum Mutual Information} $I(A;B|C) \equiv H(A|C)+H(B|C)-H(AB|C)$.
\end{itemize}
Note that the conditional quantum mutual information is positive, that is:
\begin{align}
I(A;B|C) \geq 0.
\end{align}
This equation is equivalent to the strong subadditivity inequality which will be proved in Section~\ref{StrSubAdd}.

\subsection{Conditional Quantum Entropy}
Unlike the classical conditional entropy, the quantum entropy can be negative.
This is true even the quantum entropy of single variable is never negative.
For example, in the phenomenon of entanglement, we have the joint state to be a pure state while the marginal states are mixed states.
That is,
\begin{align}
& |\psi_{AB} \rangle = \frac{1}{\sqrt{2}}(|00\rangle + |11\rangle), \nonumber \\
& \rho_{AB}= |\psi_{AB} \rangle \langle \psi_{AB}|, \nonumber \\
& \rho_A=\frac{1}{2}(|0\rangle \langle 0| + |1\rangle \langle 1|). \\
\end{align}
Therefore, the joint quantum entropy $H(AB) = 0$ while the marginal quantum entropy $H(A) = H(B) = 1$.
Hence by the definition of conditional quantum entropy, we can have the following results:
\begin{align}
&I(A;B)=2, \nonumber \\
&H(A|B)=-1. 
\end{align}
The informational statement is that we can sometimes be more certain about the joint state of a quantum system than we can be about any one of its individual parts, and this is the reason that conditional  quantum entropy can be negative.

\subsection{Coherent Information}
Negativity of the conditional quantum entropy is so important in quantum information theory that we even have an information quantity and a special notation to donate the negative of the conditional quantum entropy.
We define that the \textit{coherent information} $I(A\rangle B)_{\rho}$ of a bipartite state $\rho^{AB}$ as follows:
\begin{align}
I(A\rangle B)_{\rho} \equiv H(B)_{\rho}-H(AB)_{\rho}.
\end{align}
We can see that the quantity is the negative of the conditional quantum entropy. It is very important when we further discuss the quantum data processing and the quantum channel capacity.
\begin{subsection}{Holevo Bound}
After introducing some basic ideas of quantum information theory, we are now interested in how much information can be carried in a quantum system.
Recall that in the classical world, the amount of mutual information is bounded by the following equation:
\begin{equation}\label{eq:cBound}
I(X;Y) \leq H(X).
\end{equation}
However, in the quantum world, equation~\eqref{eq:cBound} will be further bounded by a lower limit, say {\it{Holevo bound}}.
We now give the following theorem together with an intutive example:
%and the complete proof will be written in section~\ref{sec:HolevoPf}.

\textbf{Theorem 1 (Holevo Bound)}
Assume that a sender Alice obtains a message $X$ according to some probability distribution $p_1,p_2,\cdots p_n$. She encodes the message $X$ in a quantum state $\rho_x$ and sends the quantum state to Bob. Bob receives $\{p_x,\rho_x\}$ and wants to obtain as much information as he can, then the amount of information is bounded by the following equation:
\begin{equation}\label{eq:HolevoBound}
I(X;Y) \leq H(\rho) - \underset{x}{\sum} p_xH(\rho_x) \leq H(X).
\end{equation}

We now give an example to show how {\it{Holevo bound}} works.
Suppose that Alice transmits $|0\rangle$ if $X=0$ and transmits $\cos \theta |0\rangle + \sin \theta |1\rangle$ if $X=1$, where $X=\{0,1\}$ follows a uniform distribution.
Then we can have two pure state density matrices according to the transmitted state when $X=0$ or $X=1$ as follows:
\begin{align}\label{eq:2densityMatrix}
&\rho_1 = |0\rangle \langle 0| = \begin{pmatrix} 1 &0 \\ 0 &0 \end{pmatrix}, \nonumber \\
&\rho_2 = \begin{pmatrix} \cos \theta \\ \sin \theta \end{pmatrix}
         \begin{pmatrix} \cos \theta & \sin \theta \end{pmatrix} = 
         \begin{pmatrix} \cos^2 \theta &\sin \theta \cos \theta \\
         \sin \theta \cos \theta &\sin^2 \theta \end{pmatrix}.
\end{align}
Note that the two density matrices in~\eqref{eq:2densityMatrix} both has eigenvalues $0$ and $1$; therefore, ${H(\rho_1) = H(\rho_2) = -1\log 1 - 0\log 0 = 0}$.
However, the density matrix of mixed state $\rho_1$ and $\rho_2$ is:
\begin{align}\label{eq:mixed2State}
\rho = \sum_i p_i \rho_i = \frac{1}{2}(\rho_1 + \rho_2) = 
\begin{pmatrix} 1+\cos^2 \theta &\sin \theta \cos \theta \\
 \sin \theta \cos \theta &\sin^2 \theta \end{pmatrix}.
\end{align}
The density matrix in~\eqref{eq:mixed2State} has eigenvalues $\frac{1+\cos \theta}{2}$ and $\frac{1-\cos \theta}{2}$, and hence it quantum entropy is written as: 
\begin{equation}
H(\rho) = -\frac{1+\cos \theta}{2} \log \frac{1+\cos \theta}{2} - \frac{1-\cos \theta}{2} \log \frac{1-\cos \theta}{2},
\end{equation}
which will be maximized when $\theta = \pi /2$ and the maximum equals $1$.
Therefore, by~\eqref{eq:HolevoBound}, we know that $I(X;Y) \leq H(\rho)-0 \leq 1$.
It means that Bob can obtain \textit{at most} $1$ bit of information when $\theta = \pi /2$.
This result is very important, since it essentially says that one qubit can contain at most one classical bit information.
Therefore, we can learn that quantum information does not compress classical information.
\end{subsection}

\subsection{Data proceesing}
The quantum data processing inequality is similar in spirit to the classical data processing inequality.
Recall the classical data processing case, if $X_1-X_2-X_3$ forms a Markov chain, the inequality $I(X_1;X_2) \geq I(X_1;X_3)$ holds.
Now we introduce a quantum case.

Suppose that Alice and Bob share some pure bipartite state $|\phi \rangle ^{AB}$.
The coherent information $I(A\rangle B)_{\phi}$ quantifies the quantum correlations present in the state.
Bob then processes his system B according to some CPTP map $N_1^{B\to B_1}$ to produce some quantum system $B_1$ and let $\rho ^{AB_1}$ denote the resulting state.
A condition similar to the Markov condition holds for the quantum case.
The map $N_1^{B\to B_1}$ acts only on one of Bob's systems $-$ it does not act in any way on Alice's system.
The quantum data processing inequality states that each step of quantum data processing reduces quantum correlations, in the sense that
\begin{align}
I(A \rangle B)_{\phi} \geq I(A \rangle B_1)_{\rho}
\end{align}
The inequality can be proved in Section~\ref{sec:pfDataProcess}.

%\section{Major Proofs}\label{sec_Proof}
%\begin{subsection}{Proof}
TODO
\end{subsection}

%\begin{subsection}{Proof of Holevo Bound}
TODO
\end{subsection}


\section{Conclusions}\label{sec:Conclusion}
Conclusion



\bibliographystyle{ieeetr}
\bibliography{reference.bib}

\appendix
\section{Proof of some Lemma}
\subsection{Strong subadditivity}\label{StrSubAdd}
For any quantum systems $A, B, C,$ the inequalities
\begin{align}
H(A)+H(C) \leq H(AB)+H(BC) \nonumber \\
H(ABC)+H(C) \leq H(AC)+H(BC)
\end{align}
hold.

\emph{Proof:} Define a function $T(\rho ^{ABC})$ of density operators on the system ABC,
\begin{align}
T(\rho ^{ABC}) &\equiv H(A)+H(C)-H(AB)-H(BC) \nonumber \\
&= -H(B|A) -H(B|C).
\end{align}
From the concavity of the conditional entropy we see that $T(\rho ^{ABC})$ is a convex function of $\rho ^{ABC}$ of density operators on the system $ABC$.
Let $\rho ^{ABC} = \sum_ip_i|i \rangle \langle i|$ be a spectral decomposition of $\rho ^{ABC}$.

From the convexity of $T$, $T(\rho ^{ABC}) \leq \sum_ip_iT(|i \rangle \langle i|)$. But $T(|i \rangle \langle i|)=0$ as for a pure state $H(AB)=H(C)$ and $H(BC)=H(A)$. It follows that $T(\rho ^{ABC}) \leq 0$, and thus
\begin{align}
H(A)+H(C)-H(AB)-H(BC) \leq 0,
\end{align}
which is the first inequality we set out to prove.

To obtain the second inuquality, introduce an auxiliary system $R$ purifying the system $ABC$. Then using the just-proved inuquality we have
\begin{align}
H(R)+H(C) \leq H(RB)+H(BC).
\end{align}
Since ABCR is a pure state, $H(R)=H(ABC)$ and $H(RB)=H(AC)$, so becomes
\begin{align}
H(ABC)+H(C) \leq H(AC)+H(BC).
\end{align}
Therefore, we finish the proof.
\subsection{Proof of Data Processing}\label{sec:pfDataProcess}
Suppose that Alice and Bob share some pure bipartite state $|\phi \rangle ^{AB}$ and 
Bob then processes his system B according to some CPTP map $N_1^{B\to B_1}$ to produce some quantum system $B_1$ and let $\rho ^{AB_1}$ denote the resulting state. i.e.
$\rho_{AB_1} \equiv N_1^{B\to B_1}(\phi ^{AB})$. Then the following quantum data processing inequality applies for coherent information:
\begin{align}
I(A\rangle B)_{\phi} \geq I(A\rangle B_1)_{\rho}.
\end{align}
\textit{Proof.} First consider that
\begin{align}
I(A\rangle B)_{\phi} &= H(B)_\phi -H(AB)_\phi \nonumber \\
&=H(B)_\phi \nonumber \\
&=H(A)_\phi.
\end{align}
The first equality follows by definition, the second equality follows because the entropy of a pure state vanishes, and the third equality can be proved by the Schmidt decomposition \cite{CtoQ}. Let $|\psi \rangle ^{AB_1E_1}$ be the output of the isometry $U_{N_1}^{B \to B_1E_1}$:
\begin{align}
|\psi \rangle ^{AB_1E_1} \equiv U_{N_1}^{B \to B_1E_1}|\phi \rangle ^{AB}.
\end{align}
It is also a purification of the state $\rho ^{AB_1}$. Consider that
\begin{align}
I(A \rangle B_1)_\rho &= I(A \rangle B_1)_\psi \nonumber \\
&=H(B_1)_\psi - H(AB_1)_\psi \nonumber \\
&=H(AE_1)_\psi - H(E_1)_\psi.
\end{align}
Note that $H(A)_\phi = H(A)_\psi$ because no processing occurs on system $A$. Since the quantum mutual information is always positive, the following chain of inequalities proves the objective.
\begin{align}
&I(A;E_1)_\psi \geq 0 \nonumber \\
&\rightarrow H(A)_\psi \geq H(AE_1)_\psi - H(E_1)_\psi \nonumber \\
&\rightarrow I(A\rangle B)_\phi \geq I(A\rangle B_1)_\rho.
\end{align}
The second line applies the definition of quantum mutual information, and the third line applies the results above.
%Appendix one text goes here.

%\section{Proof of some other Lemma}



\end{document}
